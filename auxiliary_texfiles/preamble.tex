\PassOptionsToPackage{unicode,pdfusetitle}{hyperref}
\PassOptionsToPackage{hyphens}{url}
\PassOptionsToPackage{dvipsnames,svgnames,x11names}{xcolor}

% Non-breakable hyphen. Use for things like "(re-)print". 
\newcommand\nobrkhyph{\mbox{-}}

% Small caps serif font, for Paper numbers, ISBN, etc
\newcommand{\I}{\textrm{\scshape i}\xspace}
\newcommand{\II}{\textrm{\scshape ii}\xspace}
\newcommand{\III}{\textrm{\scshape iii}\xspace}
\newcommand{\IV}{\textrm{\scshape iv}\xspace}
\newcommand{\V}{\textrm{\scshape v}\xspace}
\newcommand{\VI}{\textrm{\scshape vi}\xspace}
\newcommand{\VII}{\textrm{\scshape vii}\xspace}
\newcommand{\VIII}{\textrm{\scshape viii}\xspace}
\newcommand{\IX}{\textrm{\scshape ix}\xspace}
\newcommand{\X}{\textrm{\scshape x}\xspace}
\newcommand{\ISBN}{\textrm{\scshape isbn}\xspace}
\newcommand{\ISSN}{\textrm{\scshape issn}\xspace}

% Must come in the beginning. To calculate the total number of pages
\usepackage{pageslts}

% Font settings
% \usepackage[T1]{fontenc}
\usepackage[utf8]{inputenc}
% \usepackage{ebgaramond}

\usepackage[T1]{fontenc} % best for Western European languages
\usepackage[varqu,varl]{inconsolata}% a typewriter font for \mathtt
\usepackage[ebgaramond,textscale=0,semibold,vvarbb,subscriptcorrection,amsthm]{newtx}

\usepackage{microtype}

% Support for greek (non-math non-italic) text, e.g. for units like mircometer: \textmu m 
\usepackage{textgreek}

% Additional font sizes \HUGE and \ssmall, needed for figure captions
\usepackage{moresize}

\usepackage{tikz}
\usetikzlibrary{calc}

% figure captions in bold (i.e. "Figure 1" in bold), sans serif, smaller font size, hanging label, always starting on the left side
\usepackage{subfig}
% \usepackage{subcaption}
% \DeclareCaptionFont{ssmall}{\ssmall}
% \DeclareCaptionFont{tiny}{\tiny}% "scriptsize" is defined by floatrow, "tiny" not
% \captionsetup{font={ssmall,sf},labelfont={bf},format=hang,singlelinecheck=false}
\captionsetup{labelfont={bf}}
%
% % figures centred, smaller font in tables, captions on top for tables
\usepackage{floatrow}
\floatsetup[table]{position=top}
%%%%%%%%%%%%%%%%%%%%%%%%%%%%%%%%%%%% end fonts %%%%%%%%%%%%%%%%%%%%%%%%%%%%%%%%%%%%%%%%%%%%%%%%%%%%%%%

% For tables spanning the full text width
\usepackage{tabularx}

% For URLs use \url{<URL>}
\usepackage{url}

% Chapters should have numbers - a typical thesis consists of two chapters:
% One to introduce and summarize the research ("kappa"), and one for
% reproductions of the papers and manuscripts. No need to number them by default.
% If you need chapter numbers back, comment the following line.
\renewcommand{\thesection}{\arabic{section}}

% Sections and subsections have numbers, subsubsections etc do not.
\setcounter{secnumdepth}{2}

% Only chapters and sections appear in the table of contents, not subsections etc
\addtocontents{toc}{\protect\setcounter{tocdepth}{1}}

% Ensures that \cleardoublepage inserts empty pages without page number
\usepackage{emptypage}

% For text macros (e.g., small caps macros)
\usepackage{xspace}

% For "Lorem Ipsum" style place holders
\usepackage{blindtext}

% For \includegraphics
\usepackage{graphicx}

% Nice looking tables with correct spacings 
\usepackage{booktabs}

% Tables spanning more than one page
\usepackage{longtable}

% Math symbols
% \usepackage{amsfonts,amsmath,amssymb}

% University Colors
\usepackage{xcolor}
\definecolor{lublue}{RGB}{0,0,128}
\definecolor{lubronze}{RGB}{156,97,20}

% Cross-referencing
\usepackage{hyperref}
\hypersetup{
  % colorlinks = true,
  hidelinks,
  breaklinks = true,
  linkcolor  = lublue,
  filecolor  = lublue,
  urlcolor   = lublue,
  citecolor  = lubronze
}
\usepackage{cleveref}

% To include the PDFs of the papers
\usepackage{pdfpages}

% Hyphenation, support for different languages, last one is default
% Make sure to install the hyphenation packages for all languages you need
\usepackage[swedish,english]{babel}

% Paper size. Typically this works also fine when printed on A4 (text size is
% as print later, just the margins become wider to accommodate the too large
% sheets).
%% G5 format
\usepackage[paperwidth=169mm,paperheight=239mm,nomarginpar]{geometry}

% Figures can be kept in the same directory as the thesis, or in a Figure/
% directory, without need to specifying which of the two it is
\graphicspath{{Figures/}}


%%%%%%%%%%%%%%%%%%%%%%%%%%%%%%%%%%%%%%%%%%%%%%%%%%%%%%%%%%%%%%%%%%%%
% define commands for chapters, sections, and subsections that do not
% have a number, but are entered as candidates for the table of contents.
\newcommand\chap[1]{%
  \chapter*{#1}%
  \addcontentsline{toc}{chapter}{#1}
  \markboth{#1}{#1}
}
\newcommand\sect[1]{%
  \section*{#1}%
  \addcontentsline{toc}{section}{#1}
  \markright{#1}
}
\newcommand\subsect[1]{%
  \subsection*{#1}%
  \addcontentsline{toc}{subsection}{#1}
}

% Force fancyplain style into chapter pages
\usepackage{etoolbox}
\makeatletter
\patchcmd{\chapter}{\thispagestyle{plain}}{\thispagestyle{fancyplain}}{}{}
\makeatother

\usepackage{fancyhdr}

\fancypagestyle{fancyplain}{
  \fancyfoot{}
  \fancyhead{}
  \fancyhead[RO]{\thepage}
  \fancyhead[LE]{\thepage}
  \renewcommand{\headrulewidth}{0pt}
}

% References should be a section of the summary text, with number and all ...
% \renewcommand\bibsection{\section{\bibname}\markright{\bibname}}

\usepackage{enumitem}

% bibliography
\makeatletter
\def\blx@nowarnpolyglossia{}
\makeatother
\usepackage[citestyle=alphabetic,language=english,url=false,minbibnames=22,mincitenames=22,maxbibnames=22]{biblatex}
\addbibresource{phdthesis.bib}
